\section{Clustering}
Beim Clustering geht es darum, eine möglichst natürliche 
Partitionierung der Daten in \(k\) Cluster zu finden. Cluster
können anhand ihrer \textbf{Intra-} und \textbf{Inter-Cluster
Similarity} qualifiziert werden, also wie sehr sich die Elemente
im selben Cluster ähneln und wie sehr sich die Objekte aus
verschiedenen Clustern unterscheiden. Gemessen wird diese
Ähnlichkeit im Allgemeinen mit der Distanz der Objekte zueinander
gemessen. Damit lässt sich die so genannte \textbf{Criterion
Function} aufstellen, die minimiert werden soll:
\[\text{min } E = \sum_{i=1}^k \sum_{\vec{x}\in C_i} 
d(\vec{x},\ \vec{m_i}).\]
Hierbei muss \(k\) jedoch bekannt sein; das Problem ist nun
jedoch die Wahl von \(k\). Das beste Ergebnis entsteht
trivialerweise, wenn man jedem Datenobjekt seinen eigenen
Cluster zuweist, wodurch \(E\) verschwindet. Das kann
jedoch nicht gewünscht sein. 

Die Wahl der Anzahl der Cluster ist also generell ein nicht 
offensichtlich lösbares Problem. Es würde ein profundes Wissen
über die Verteilung der Daten voraussetzen, was jedoch meistens
nicht vorhanden ist. Deswegen werden stattdessen oft verschiedene
Clusterings erzeugt, die dann miteinander verglichen werden. Auf
das beste Clustering fällt dann die Wahl.

Aber was heißt hier das "'beste"'? Wie kann man die Güte eines
Clusterings qualifizieren? Abhilfe verschafft uns hier der 
\textbf{Silhouette-Koeffizient}. Dieser nutzt zum einen die
Eigenschaften von Intra-/ Inter-Cluster Similarity aus und 
wächst zum anderen nicht monoton mit \(k\), was bei der
Criterion Function nicht gegeben ist. Die Silhouette eines Objekte
soll ausdrücken, wie sehr das Objekt tatsächlich zum
theoretisch angenommenen Cluster und nicht zu nächst gelegenen
gehört. Hierbei seien nun
\begin{align*}
a(o)&=\frac{1}{\left| C(o) \right|}\sum_{p \in C(o)}dist(o,\ p),\\
b(o)&=\min_{C_i \neq C(o)} \left(\frac{1}{\left| C_i \right|}
\sum_{p \in C_i}dist(o,\ p)\right)
\end{align*}
die mittleren Distanzen von Objekt \(o\) zu den Objekten des
eigenen Clusters (\(a(o)\)) und den Objekten des nächstgelegenen
anderen Clusters (\(b(o)\)). Das gewählte Objekt \(o\) kann
nun entweder tatsächlich ein Objekt der Menge sein, oder ein
beliebiger Repräsentant (z.B. der Centroid, dazu später mehr).

Die Silhouette von \(o\) ist nun
definiert als
\[ s(o) = \begin{cases}
		0 & \text{falls } a(o)=0\text{, d.h. } \lvert C_i \rvert = 1,\\
		\frac{b(o)-a(o)}{\max\{a(o),\ b(o)\}} & \text{sonst.}
		\end{cases}\]
Der Wertebereich von \(s(o)\) erstreckt sich auf \([-1,\ 1]\) .
Werte nahe 1 bedeuten, dass unser Clustering relativ gut ist,
da die Objekte in ihren eigenen Clustern liegen, womit \(a(o)\)
klein wird. Liegen hingegen die Objekte in den falschen Clustern, 
dann wird \(b(o)\) sehr klein, \(a(o)\) jedoch sehr groß, womit
\(s(o)\) negativ wird. Es sind also Werte nahe bei 1 für \(s(o)\)
wünschenswerte. Möchte man nun ein Clustering \(\mathcal{C}=
\{C_1,\ \dots ,\ C_k\}\) als Ganzes bewerten, so kann man
\[ silh(\mathcal{C}) = \frac{1}{\lvert \mathcal{C}\rvert}
\sum_{C_i \in \mathcal{C}}\frac{1}{\lvert C_i\rvert}
\sum_{o\in C_i} s(o)\]
berechnen. Durch die Normalisierungen erstreckt sich der
Wertebereich von silh(\(\mathcal{C}\)) wieder auf  \([-1,\ 1]\).
Über die Interpretation, ab wann nun ein Clustering als gut
zu bezeichnen ist, kann man, wie bei fast allen Koeffizienten,
im Detail streiten. Grob sagt man, dass ein Clustering für 
Werte \(> 0,5\) mittelmäßig bis gut sind, für Werte zwischen
\(0\) und \(0,5\) keine oder nur eine schwache Strukturierung 
vorliegt, und für negative Werte unser Clustering wohl falsch ist.

Um diese Einführung in das Konzept von Clustering zu beenden 
und mit den konkreten Clustering Algorithmen fortfahren zu können,
müssen wir noch kurz ansprechen, wie man Distanzen zwischen
Mengen von Datenobjekten misst. Dafür werden wir die drei folgenden
Methoden verwenden, die im Wesentlichen gleich funktionieren,
nämlich
\begin{align*}
\text{Single Link: }& dist_{sl}(X,\ Y) &= \min_{x\in X,\ y\in Y} 
dist(x,\ y),\\
\text{Complete Link: }& dist_{cl}(X,\ Y) &= \max_{x\in X,\ y\in Y} 
dist(x,\ y),\\
\text{Average Link: }& dist_{al}(X,\ Y) &= 
\frac{1}{\lvert X\rvert \lvert Y\rvert}\sum_{x\in X,\ y\in Y} 
dist(x,\ y).
\end{align*}
Der Single Link ist der Abstand zwischen den nächsten beiden Objekten
zweier Cluster, der Complete Link beschreibt die Distanz zwischen den
beiden am weitesten voneinander entfernten Objekten. Das Problem
dieser beiden Maße ist, dass sie für Ausreißer besonders anfällig sind.
Der Average Link ist in dieser Hinsicht geeigneter, insbesondere, wenn
wird viele Datenobjekte zur Verfügung haben.


Grob lassen sich Clustering Verfahren in folgende
Kategorien einteilen:
\begin{itemize}
\item \textbf{Partitionierend:} Hierbei werden die Daten in
Partitionen eingeteilt, so dass jede Partition nicht leer ist und jedes
Objekt zu \textit{genau einem} Cluster gehört. Meistens gehen
die Verfahren derart vor, dass zunächst ein initiales Clustering erstellt
wird, welches in iterativen Abläufen durch Verschieben der Objekte
in andere Cluster verbessert werden soll. Durch die schiere Menge
an möglichen Partitionierungen ist es selten möglich, ein globales
Optimum zu finden, insbesondere wenn wir es mit großen
Datenbeständen zu tun haben. Stattdessen greift man auf Heuristiken
zurück, indem z.B. Cluster durch Medoide (dazu gleich mehr) oder
ihren Mittelwert repräsentiert werden. Diese Verfahren sind gut in der
Lage, sphärische Cluster in kleinen bis mittleren Datenbeständen zu 
finden. Für große Datenmengen oder komplex Clusterformen werden
Erweiterungen der vorgestellten Algorithmen benötigt.
\item \textbf{Hierarchisch:} Hierbei geht es darum, eine Hierarchie
aufzubauen, die unseren Datenbestand widerspiegelt. Im Wesentlichen
unterscheidet man hier \textbf{agglomerative} (bottom-up) und
\textbf{divisive} (top-down) Methoden. Das Problem bei diesen 
Verfahren ist, dass man einen \textit{split} oder \textit{merge} 
Schritt nicht mehr rückgängig machen kann. Das hat zwar den Vorteil,
dass die kombinatorische Zahl an theoretisch zu betrachteten
Möglichkeiten entfällt, dafür jedoch Fehler auch nicht mehr rückgängig
gemacht werden können. Es gibt verschiedene Lösungsansätze, von
denen wir uns später BIRCH ansehen werden, in welchem zunächst
mittels agglomerativen hierarchischem Clustering ein (guter) 
Ausgangspunkt für ein partitionierendes Verfahren erzeugt wird (man
möchte hier vor allem die iterative Reallokation der Datenobjekte in
den partitionierenden Verfahren nutzen).
\item \textbf{Dichte-basiert:} Hierbei wird die Objektdichte
in bestimmten Teilräumen betrachtet. Dadurch ist es möglich, zum
einen Noise möglichst aus unseren Clustern rauszuhalten, was in
den anderen Verfahren nicht ohne weiteres gegeben ist, und zum 
anderen beliebige Formen von Clustern zu finden. 
\item \textbf{Probabilistisch:} Bei diesen Verfahren werden
Objekte nur mit einer bestimmten \textit{Wahrscheinlichkeit} einem
bestimmten Cluster zugewiesen. Diese Verfahren nehmen für jedes
Cluster eine Verteilungsfunktion an, mithilfe derer die Zugehörigkeit
eines Objektes zu besagtem Cluster gemessen werden kann. Sie bieten
auch die Möglichkeit, anhand der Daten die Anzahl der Cluster zu 
bestimmen, auch unter Berücksichtigung von Noise und Ausreißern.
\end{itemize}
Verfahren aus verschiedenen Kategorien lassen sich auch leicht
miteinander kombinieren (siehe BIRCH).

\subsection{Partitionierendes Clustering}
\subsubsection{k-Means}
\textbf{k-Means} ist ein ziemlich einfaches, naives Clustering
Verfahren. Am Anfang wählt man beliebig \(k\) Datenobjekte,
die als Surrogate für unsere Cluster dienen, und weist dann jedes
andere Datenobjekt dem ihm am nächsten gelegenen Surrogat zu.
So erhält man eine Partitionierung aller Datenobjekte in \(k\) 
Cluster. Da man nicht davon ausgehen kann, dass man beim ersten
Mal bereits die optimalen Mittelpunkte der Cluster gefunden hat,
geht man iterativ wie folgt vor: Man bestimmt den Mittelpunkt eines
jeden Clusters und wählt im nächsten Schritt diese Mittelpunkte als
Surrogate für die Neuzuweisung der Datenobjekte. Dieses Vorgehen
verschiebt in jedem Schritt den Mittelpunkt des aktuellen Clusters
in Richtung des Mittelpunktes des (lokal) optimalen Clusters, so lange,
bis für eine bestimmte Anzahl an Iterationen keine signifikante 
Verbesserung mehr möglich ist (z.B. 4 Iterationen lang keine
Verbesserung \(> \varepsilon = 10^{-4}\)). Das Gütemaß kann
hierbei sowohl die Criterion Function als auch der Silhouette Koeffizient
sein, da unser \(k\) in jeder Iteration gleich bleibt. Die Wahl des
Mittelpunktes kann auch auf zwei mögliche Methoden geschehen.
Entweder wählt man den \textbf{Medoid} des Clusters oder den
\textbf{Centroid}. Der Medoid ist der Datenpunkt, der dem 
tatsächlichen Clustermittelpunkt, dem Centroid, am nächsten ist.
Centroide sind also nicht notwendigerweise auch tatsächlich in unserem
Datenbestand enhalten, jedoch kann der Fall eintreten, dass Medoid 
und Centroid zusammenfallen.

Die Wahl der Punkte in der Initialisierungsphase (auch \textbf{Seeds}
genannt) ist maßgebend für das spätere Ergebnis. k-Means erzeugt
nämlich zunächst nur lokale Optimalpunkte. Ein ungeschickte oder
unglückliche Wahl der Seeds kann dazu führen, dass man ein Lösung
findet, die vom globalen Optimum weit entfernt ist, dieses jedoch
nie mit den gegebenen Seeds erreichen kann. Dies könnte z.B. dann passieren, 
wenn die Seeds dicht beieinander liegen. Ein Ansatz wäre
eine Greedy-Lösung, indem die Seeds immer mit einem Mindestabstand
zueinander gewählt werden müssen. Dadurch wird eine bessere 
Abdeckung der Raumes gewährleistet. Insgesamt ist diese Lösung aber
auch kein Garant für ein (nahezu) globales Optimum, weswegen man
auch mit einem Greedy-Ansatz nicht vermeiden kann, mehrere Runs
von k-Means laufen zu lassen mit unterschiedlichen Seeds.

Dies Stellt bereits eine der Variationen für k-Means dar. Neben der 
Anzahl an Runs kann auch das Stop-Kriterium angepasst werden,
wodurch die Genauigkeit der Lösung variiert werden kann. Hat man
es mit sehr großen Datenbeständen zu tun, so kann es zu teuer sein,
in jeder Iteration die ganze Datenbank zu scannen. Hierfür ist Sampling
eine Lösung. Generell lässt sich aber wohl sagen, dass k-Means 
besonders empfänglich für eine raffinierte Wahl der Seeds ist, da
damit nicht nur die Laufzeit, sondern auch die Güte der Lösung
stark beeinflusst werden können.

Eine Variation des Verfahrens, die sich nicht mit den Parametern
von k-Means beschäftigt, ist das Ersetzen von womöglich schlechten
Medoiden. Ein Medoid kann als schlecht angenommen werden, wenn er
weniger als \(\frac{N}{k} \cdot minDev\) Datenobjekte zugewiesen
bekommen hat. Hierbei ist \(N\) die Anzahl der Datenobjekte
insgesamt, \(k\) die Anzahl der Cluster und \(minDev\) eine 
Konstante, die der Tatsache Rechnung trägt, dass es wohl kaum eine
Gleichverteilung der Daten auf alle Cluster gibt. Je nach Wahl von
\(minDev\) sind also auch kleine Cluster möglich.

\subsubsection{k-Means in der Variante CLARANS}
Wie bereits eben gesagt, ist k-Means für sehr große Datenbestände
ungeeignet, da zu viele Datenobjekte in jedem Schritt 
zu oft betrachtet werden
müssten. "'Zu oft"' bedeutet hier, dass in jeder Iteration jeder Medoid
verändert wird, d.h. man muss alle Datenobjekte mit allen Medoiden
neu vergleichen. Ebenfalls wurde gesagt, dass Sampling hier helfen kann.
Dieses Verfahren hat sogar einen Namen, nämlich \textbf{CLARA}
(Clustering LARge Applications). CLARA nimmt sich ein Sample,
und führt auf diesem k-Means aus. Aufgrund der kleineren Größe
kann k-Means effektiv ausgeführt werden und bei einer hinreichend 
randomisierten Auswahl des Samples ist das Ergebnis meist auch
recht gut.

Ein Problem stellt jedoch die Beschränkung auf ein einziges Sample
dar. CLARA wählt ein Sample, was effektiv eine Einschränkung des
Raumes bedeutet. Ist in diesem Raum nun nicht der Optimalpunkt
eines Clusters enthalten, so kann CLARA nie ein optimales Ergebnis 
liefern. Man müsste CLARA mehrmals mit verschiedenen Samples
laufen lassen.

Dieses Problem wird durch \textbf{CLARANS} (Clustering Large
Applications based upon RANdomized Search) gelöst. Die Grundidee
ist, dass man einen Graphen betrachtet, in dem die Knoten Seeds
für k-Means darstellen. Knoten sind genau dann verbunden, wenn
sich die Seeds in \textit{genau einem} Medoiden unterscheiden.
Der Unterschied zu CLARA besteht nun darin, dass man nicht auf einem
Sample des Datenbestandes arbeitet, sondern auf einem Sample aus
Nachbarn: Man wählt zunächst einen beliebigen Seed, also einen 
beliebigen Knoten aus dem Graphen aus. Danach wählt man beliebig eine
zuvor festgelegte Anzahl an Nachbarknoten und vergleicht die Clusterings
untereinander. Das Clustering, welches am besten ist, wird als aktuelles
Minimum gewählt. Ebenso wie k-Means findet CLARANS zunächst nur 
lokale Optima, weswegen mehrere Runs von CLARANS mit unterschiedlichen
Startseeds nötig sind. 

Experimentell konnte gezeigt werden, dass CLARANS schneller läuft,
als der naive k-Means und CLARA. Für die Performance kann man
verschiedene Variationen bei der Auswahl der Nachbarn sowie der
zugrunde liegenden Datenstruktur (z.B. R-Baum etc.) wählen. Das im
letzten Abschnitt Gesagte über das Ersetzen von schlechten Medoiden
lässt sich auch auf CLARANS übertragen.

\subsubsection{BIRCH}
Whut?\footnote{Okay, es lässt sich darüber streiten, ob wir BIRCH als
hierarchisches
oder als partitionierendes Verfahren einstufen. Let's be honest: BIRCH
heißt \textit{Balanced Iterative Reducing and Clustering using Hierarchies}
-- das trägt es ja schon im Namen. In der Vorlesung wird es als
partitionierendes Verfahren vorgestellt; das könnte daran liegen, dass
am Ende doch nur ein Seeds für ein partitionierendes Verfahren entsteht 
bzw. dass die Lösung, die BIRCH anfangs produziert aus diversen
Gründen weniger handlich ist, als es ein auf k-Means basierendes Verfahren
liefern würde.}

\noindent BIRCH benutzt als Grundgerüst einen so genannten
\textbf{CF-Tree} (Clustering Feature Tree). In diesem werden die
Cluster anhand ihrer bestimmten Features bestimmt und das auf
mehreren Hierarchieebenen. Die Features eines Clusters sind ein
Tripel \(CF=(N,\ \vec{LF},\ SS)\), wobei \(N\) die Zahl der 
Datenobjekte im Cluster, \(\vec{LF} = \sum\nolimits_{i=1}^N
\vec{o_i}\) die lineare Summe aller Objekte des Clusters und 
\(SS=\sum\nolimits_{i=1}^N
\vec{o_{i}^2}\) die Summe der Quadrate der Clusterobjekte ist.
Mit diesen Werten lassen sich weitere Eigenschaften von Clustern
berechnen, z.B. der Centroid und der Radius.\footnote{Sollte sich der
Verfasser dieser Zeilen nicht irren, dann gilt
für den Radius \(R = \left(\frac{SS}{N}-\frac{LS^2}{N^2}\right)^{\frac{1}{2}}\).}
Diese kompakte 
Darstellung von Clustern macht den CF-Tree zu einem sparsamen
Repräsentanten des Clusterings auf unserem Datenbestand. 

Der Vollständigkeit halber hier noch die Definitionen von
\begin{align*}
&\text{Centroid:} &\vec{x_0} = \frac{\sum_{i=1}^N \vec{x_i}}{N},\\
&\text{Radius:} &R =\left( \frac{\sum_{i=1}^N \left(\vec{x_i}-\vec{x_0}
\right)^2}{N}\right)^{\frac{1}{2}},\\
&\text{Durchmesser:} &D = \left( \frac{\sum_{i=1}^N \sum_{j=1}^{N} \left(\vec{x_i}-\vec{x_j}
\right)^2}{N(N-1)}\right)^{\frac{1}{2}}\\
\end{align*}
\begin{align*}
&\text{und der Durchschnittlichen Inter Cluster Distanz:} &\quad\\
 &D_2 = \left( \frac{\sum_{i=1}^{N_1} \sum_{j=N_1 +1}^{N_1 + N_2} \left(\vec{x_i}-\vec{x_j}
\right)^2}{N_1 N_2}\right)^{\frac{1}{2}}. &
\end{align*}

Kommen wir nun aber endlich zum tatsächlichen CF-Tree. Der CF-Tree ist
ein höhenbalancierter Baum, in dem jeder Knoten einem Cluster entspricht. 
In den Blättern sind die s.g. \textbf{Elementarcluster} enthalten, in den 
inneren Knoten befinden sich Mengen an Tupeln der Form \((CF_i,\ \text{
child}_i)\). Dabei enthält Child nur einen Verweis auf den Knoten, dessen 
Cluster aggregiert das CF des aktuellen Childs begründen (geht man im
Baum nach oben, so ergibt sich das CF eines Knotens aggregiert aus den
CF's seiner Children). Die Parameter, die man anfangs bei der Erstellung
des Baumes mitgeben muss sind
\begin{itemize}
\item \textbf{B}: Der Fan-Out für innere Knoten, also die Zahl der Kinder,
die ein innerer Knoten maximal haben darf.
\item \textbf{B'}: Die Kapazität eines Blattes.
\item \textbf{T}: Der Schwellenwert für den Radius oder Durchmesser
eines \textbf{Elementarclusters}.
\end{itemize}
Der Parameter T ist wichtig, da ohne ihn beim Einfügen der Elemente
in den Baum ansonsten ein einziger Megacluster entstehen würde.

Das \textit{Einfügen} in den Baum erfolgt relativ einfach. Es werden
nacheinander alle Datenobjekte in den Elementarcluster eingefügt, dessen
Centroid den kleinsten Abstand zum einzufügenden Datenobjekt hat. Wird
dabei T überschritten, so muss ein neuer Elementarcluster erzeugt werden.
Wird damit jedoch die maximale Anzahl an Clustern pro Blatt (B') überschritten
so muss der Knoten in zwei neue Blattknoten gesplittet werden.

Das \textit{Splitten} funktioniert, indem man die beiden Cluster im Knoten
als Seeds für die neuen Blätter wählt, die am weitesten voneinander entfernt
sind. Die übrigen Cluster werden dann dem  Knoten des näher gelegenen
Seeds zugewiesen. Danach sowie nach jedem eingefügten Datenobjekt müssen
die Knoteneinträge bzgl. ihrer CF nach oben hin bis zur Wurzel geupdated
werden. Das Splitting erzeugt zwei neue Blattknoten, d.h. der Elternknoten
des gesplitteten Knotens muss nun zwei Einträge aufnehmen, anstelle eines
einzigen. Ist der Elternknoten jedoch ebenfalls bereits voll, dann muss auch
dieser gesplittet werden, wodurch sich für dessen Eltern das gleiche Problem
ergeben kann.

Damit der CF-Tree nicht komplett wild wächst und Elementarcluster, die
eigentlich nahe beieinander liegen durch ungünstiges Splitting im Baum weit
voneinander entfernt stehen, können ähnliche Cluster auch gemergt werden.
Ein \textit{Merge} kann sich insbesondere nach einem Splitting anbieten.
Führt man die eben genannte "'Splitting-Kette"' ganz bis zum Schluss durch,
so gibt es einen Knoten, der nicht mehr gesplittet wurde (das kann auch eine
etwaige neue Wurzel sein). Man betrachtet nun die Kinder dieses Knotens und
wählt das Paar, das sich am nächsten ist. Dies sind Kandidaten für einen Merge.
Sind beide Kandidaten nicht das Ergebnis des Splittings (wenn einer der beiden
ein "'Split"'-Knoten ist, ist das noch in Ordnung), dann werde diese beiden
zu einem neuem Knoten zusammengefügt. Gegebenenfalls sind nach dem Merge
noch weitere Splittingschritte notwendig. Dafür gewinnt man jedoch, dass
die Wahrscheinlichkeit, dass ähnliche Cluster in komplett verschiedenen 
Subtrees des Baumes landen, verringert wird.

Dies zeigt ein deutliches Problem von BIRCH auf, was es als alleiniges 
Clustering disqualifiziert. Das Splitting von Knoten beruht nicht auf der Struktur
des zugrundeliegenden Datenbestandes. Stattdessen wird lediglich aufgrund 
von technischen Eigenheiten des Algorithmus und des Rechnersystems
gesplittet. Auch sind die gefundenen Cluster sehr abhängig vom gewählten
T. Hat man z.B. in seinen Daten ein eindeutiges, schweres und dichtes natürliches
Cluster, das jedoch einen Radius größer T hat, dann kann dieses nie identifiziert
werden. Stattdessen wird es im Baum durch einzelne Elementarcluster 
repräsentiert, die im schlimmsten Fall sogar in verschiedenen Subtrees stehen.
Damit ist zum einen das direkte Auslesen der Cluster aus dem Baum keine gute Idee 
und zum anderen ist die Aussagekraft der Hierarchischen Struktur des
Baumes wenigstens fragwürdig. Deswegen haben auch die Erfinder von BIRCH
vorgeschlagen, dass man stattdessen das Ergebnis von BIRCH als Seeds für
ein anderes Clustering Verfahren nimmt, in der Hoffnung, dass durch die
Vorarbeit von BIRCH ein gutes Ergebnis erzielt werden kann.

\subsection{Hierarchisches Clustering}
Das Prinzip von Hierarchischem Clustering dürfte jetzt spätestens seit BIRCH
bekannt sein. Wie bereits in der Einführung erwähnt unterscheidet man
im Wesentlichen zwischen zwei Ansätzen: den \textbf{agglomerativen} und
den \textbf{divisiven} Verfahren. Das Ergebnis jeweils lässt sich
gut als Dendogramm darstellen, da so die Verschachtelung der Cluster gut 
visualisiert wird.

Ein einfaches agglomeratives Verfahren ist das iterative Zusammenfügen von
einzelnen Objekten zu immer größeren Clustern basierend auf deren Abstand.
Dies funktioniert im Prinzip genauso wie das Bestimmen eines minimalen
Spannbaumes. Der Rechenaufwand entsteht in diesem Verfahren vor allem
zu Beginn, da am Anfang die Abstände aller Objekte jeweils zueinander 
bestimmt werden müssen. In den Folgenden Schritten müssen dann
lediglich für das neue Cluster die Entfernungen berechnet werden.
Insgesamt gibt es also
eine quadratische Laufzeit (\(\frac{n(n-1)}{2}\) um genau zu sein)
in der Größe des Datenbestandes. Nicht effizient, aber 
es funktioniert. Ein Vorteil ist, dass durch dieses Vorgehen auch lokale Muster
gut berücksichtigt werden können.

Ein Beispiel für ein divisives Clustering wäre \textbf{DIANA} (DIvisiv ANAlysis).
Hierbei hat man anfangs einen "'Mega Cluser"', der alle Datenobjekte beinhaltet.
Danach bestimmt man das Objekt, dass im Mittel den größten Abstand zu allen
anderen Objekten hat. Dieses Objekt wird als Seed für unsere "'SplinterGroup"'
gewählt. Danach wird iterativ jedes übrige Objekte aus dem Rest (also Cluster
ohne SplinterGroup) auf seine Clusterzugehörigkeit geprüft. Dafür wird die
durchschnittliche Unterschiedlichkeit zu jedem Cluster berechnet und daraus
die Differenz gebildet. Formal ergibt sich
\[D(o') = \sum_{o_j \in\, \text{Rest}} \frac{d(o',o_j)}{\lvert \text{Rest}
\rvert} - \sum_{o_j \in\, \text{SplinterGroup}} \frac{d(o',o_j)}{\vert
\text{SplinterGroup} \rvert}.\]
Man beachte die Ähnlichkeit zur Silhouette eines Objektes. Prinzipiell wird bei
beiden ja das Gleiche geprüft, hier findet lediglich keine Normierung statt.
Hat man nun für alle übrigen Objekte \(D(o)\) berechnet, so wählt das Objekt
mit dem größten positiven Wert aus, und fügt dieses der SplinterGroup hinzu. 
Danach wird das Spiel so lange wiederholt, bis kein Objekt mehr ein positives 
\(D(o)\) besitzt. Danach ist man mit diesem Cluster fertig. Als nächstes wählt
man den Cluster, der den größten Durchmesser hat und verfährt genauso mit 
diesem. Man kann den Algorithmus entweder so lange laufen lassen, bis man
tatsächlich alle Datenobjekte in eigene Cluster unterteilt hat, oder man legt
stattdessen einen unteren Grenzwert fest, der Cluster ab einer gewissen Dichte
und Größe als fest annimmt und diese nicht mehr splittet (d.h. findet man durch
Splitting ein kleines, sehr dichtes Cluster, dann macht es wohl keinen Sinn, 
dieses weiter aufzuteilen). DIANA war einer der ersten (vielleicht sogar 
\textbf{der} erste) Ansatz eines divisiven Verfahrens, der nicht an der
exponentiellen Anzahl an Splittmöglichkeiten am Anfang des Algorithmus
gescheitert ist. Hat man nämlich am Anfang nur sein Mega Cluster gegeben, ist
es nicht trivial ersichtlich, wie man dieses nun splittet. Würde man nun alle
Möglichkeiten miteinander vergleichen wollen, dann müsst man insgesamt
\(2^{n-1}-2\) Möglichkeiten betrachten. DIANA umgeht dieses Problem durch
sein iteratives Vorgehen. Ein Vorteil von Hierarchischen Verfahren insgesamt ist,
dass sie sich nicht zu sehr auf den "'Kleinscheiß"' verfahren; durch den top-down
Ansatz sind globale Trends in den Daten besser erfassbar, als bei 
agglomerativen Verfahren.

\subsection{Probleme mit hochdimensionalen Räumen}
Wie bereits in früheren Kapiteln erwähnt ist die Anzahl der Dimensionen unseres
Merkmalraumes ein gravierendes Problem. Haben wir Datensätze mit vielen 
hundert Dimensionen, dann kann wohl kaum der Fall eintreten, dass sich die
Daten gut in Cluster einteilen lassen. Stattdessen hat man mehr einen Raum
voller Noise. Dichte-basierte Verfahren kann man in den Wind schießen und
Hierarchien in so vielen Dimensionen sind auch wenig sinnvoll.

Es kann jedoch der Fall sein, dass es Datensätze gibt, die nur in einer kleinen
Anzahl an Dimensionen ein Cluster bilden. Diese lassen sich durch das so
genannte \textbf{Projected Clustering} finden. Dieser Algorithmus erhält
als Input \(k\), die Anzahl der zu findenden Cluster, und \(l\), die
durchschnittliche Zahl der Dimensionen pro Cluster. Als Output erhält man
eine Partitinierung der Daten in \(k+1\) Mengen (+1 für die Daten, die in
kein Cluster gefallen sind) und eine Menge an Dimensionen \(D_i\) für jedes
Cluster \(C_i\).

Projected Clustering funktioniert im Prinzip genauso wie k-Means, nur dass
nun am Anfang jedes Schrittes die Dimensionen, in denen die Medoiden 
signifikant sind, ermittelt werden müssen. 

Dies geschieht dadurch, dass man
für jede Dimension einzeln den durchschnittlichen Abstand zwischen Medoid
\(m_i\) und den Objekten in seiner \textbf{Locality} \(\mathcal{L}_i\)
bestimmt. Die Locality kann man sich als Kugel um den Medoiden vorstellen.
Die Größe dieser Kugel kann beliebig gewählt werden, in der Vorlesung wurde
sie mit dem Abstand zum jeweils nächsten Medoiden vorgestellt. Die berechnete
durchschnittliche Distanz der Objekte in der Locality
 in einer bestimmten Dimension \(j\) sei als \(X_{i,j}\) bezeichnet.
Das \(i\) steht für die Locality \(\mathcal{L}_i\). Hat man alle \(X_{i,j}\)
einer Locality berechnet, so berechnet man aus diesen nun wieder den 
Durchschnitt \[Y_i = \frac{\sum_{j=i}^d X_{i,j}}{d},\] wobei \(d\) die
Gesamtzahl der Dimensionen ist. Spontan lässt sich festhalten, dass, falls 
die Differenz
\(X_{i,j}-Y_i\) negativ wird, die Dimension \(j\) wichtig für den Medoiden
\(m_i\) ist. In der Vorlesung wurde ein Beispiel gezeigt, in dem ein Cluster
zunächst nur in einer Dimension signifikant war. Hat man jedoch die Daten
des Clusters in allen Dimensionen \textbf{gestaucht}, so wurden dadurch 
auch absolut 
gesehen die \(X_{i,j}\) kleiner. Verglichen mit der Ausgangsverteilung der
Daten wären diese nun auf einmal in allen Dimensionen signifikant! Um solche
Effekte zu vermeiden, benutzen wir stattdessen die Standardabweichung der
\(X_{i,j}\) von ihrem jeweiligen \(Y_i\), um die Werte zu normalisieren.
Die Standardabweichung sei 
\[\sigma_i =\sqrt{\frac{\sum_j (X_{i,j} - Y_i)^2}{d-1}}.\]
Damit lässt sich nun die Kennzahl \(Z_{i,j} = \frac{X_{i,j}-Y_i}{\sigma_i}\) 
berechnen. Wenn die Standardabweichung klein wird, dann wird der Quotient 
insgesamt groß. Das heißt, wenn die Daten in vielen Dimensionen ähnlich um
\(m_i\) verteilt sind, scheint unser Medoid ein brauchbarer Mittelpunkt für
seine Locality und damit potentiell auch für sein Cluster zu sein. Deswegen 
werden die \(Z_{i,j}\) zunächst einmal der Größe nach sortiert. Danach wählt
man die ersten \(k \cdot l\) \(Z_{i,j}\) aus und ordnet diese ihren Medoiden zu.
Der Sinn dahinter ist, dass man die Dimensionen jeweils für jeden Medoiden 
anhand ihrer Signifikanz für die Beschreibung der Locality des Medoiden sortiert.
Und das sogar Locality- und damit auch Cluster-übergreifend. Damit kann man 
die \textit{wichtigsten Dimensionen für die wichtigsten Medoiden} unseres
Datenbestandes finden. Auch ist durch diese Wahl der \(Z_{i,j}\) die 
Bedingung, dass es pro Cluster im Durchschnitt \(l\) Dimensionen gibt,
immer noch gewährleistet.

Nachdem wir nun den Medoiden ihre Dimensionen zugeordnet haben, müssen
wir wieder die Daten den Medoiden zuordnen. Offenbar sollte dies am besten
die Dimensionen, in denen die Medoiden signifikant sind, berücksichtigen.
Dabei haben wir jedoch das Problem, dass gegebenenfalls Medoiden mit weniger
Dimensionen einen Vorteil bei der Distanzbestimmung haben. Deswegen 
verwenden wir die \textit{Manhattan Segmental Distance}, die jedes mal
die Anzahl der Dimensionen pro Medoiden herausnormalisiert.

Haben wir nun alle Datenobjekte ihrem nächsten Medoiden zugeordnet, dann
berechnen wir den Schwerpunkt dieser Partition und der Spaß geht von vorne los.


\subsection{Clustering mit kategorischen Daten}
Wenn wir uns unser Warenkorbszenario anschauen, dann bestehen unsere
Transaktionen nicht aus numerischen Werten, sondern sind Bit-Vektoren, die
angeben, ob ein Item enthalten ist oder nicht. Bei einer hohen Zahl an
möglichen Items ist zudem auch die Dimension unserer Vektoren sehr hoch.
Dies hat zur Folge, dass selbst Kunden mit einem eigentlich gleichem Kaufverhalten
unterschiedlich eingestuft werden, z.B. Premium-Kunden, die zwar Luxusgüter
kaufen, aber eben nicht \textbf{exakt} die selben. Außerdem funktionieren
herkömmliche euklidische Distanzmaße nicht sinnvoll, z.B. wird ein Bit-Vektor,
der ganz am Anfang eine 1 und ansonsten nur 0 enthält ähnlich eingestuft
wie ein Vektor, der ganz am Ende eine 1 und ansonsten nur 0 hat. Die beiden
Vektoren stimmen nur in den Items überein, die nicht gekauft wurden. Das
ist nicht unbedingt hilfreich (außer in Sonderfällen evtl.).

Deswegen brauchen wir ein anderes Ähnlichkeitsmaß, nämlich den
so genannten \textbf{Jaccard-Koeffizienten}. Dieser beschreibt die
Ähnlichkeit zweier Transaktionen anhand ihrer gleichen Items.
\[\text{Jaccard } = \frac{\left| T_1 \cap T_2 \right|}{\left| T_1 \cup T_2 \right|}\]
Der Jaccard-Koeffizient alleine reicht aber noch nicht aus. Wie im Beispiel
aus der Vorlesung gezeigt, funktioniert der Jaccard-Koeffizient als Distanzmaß
in einem naiven agglomerativen Verfahren deswegen nicht, da die Transaktionen
nur paarweise betrachtet werden. 

Es lässt sich feststellen, dass, um bei unseren beiden Premium-Kunden zu bleiben,
diese zwar jeweils für sich genommen nur wenig gleiches Items in ihren
Transaktionen haben; wenn man jedoch noch andere hinzunimmt, so
kann man feststellen, dass sie mit diesen anderen mehr Ähnlichkeit haben.
Über diese "'Brücke"' sind die beiden Ausgangskunden dann miteinander
verbunden. Dieses Konzept der \textbf{Nachbarschaft} kann uns beim
Clustering behilflich werden. 

Dafür gibt es die \textbf{Link-basierte Methode}. Bei ihr werden zwei
Punkte dann als Nachbarn betrachtet, wenn ihre Ähnlichkeit einen
gewissen Schwellwert überschreitet. Was genau dieses Ähnlichkeitsmaß ist,
ist abhängig von der Situation. Bei unseren Transaktionen, also Mengen von
Items, bietet sich der Jaccard-Koeffizient an. Nun werden Links zwischen den
einzelnen Punkten erzeugt, wobei die Anzahl der Links zwischen zwei Punkten
der Zahl ihrer gemeinsamen Nachbarn entspricht. Als nächstes werden dann
die beiden Punkte / Cluster gemergt, die die größte Linkzahl zwischen sich 
haben. Dies wird so lange wiederholt, bis man nur noch einen einzigen Cluster
hat (also ein agglomeratives hierarchisches Verfahren).

Zusammenfassend lässt sich festhalten, dass Clustering mit kategorischen
Daten nicht ohne weitere Denkarbeit zu lösen ist. Man muss sich geeignete
Kriterien überlegen, der Ansatz mit der Nachbarschaft ist aber meistens
schon eine recht gute Wahl und liefert brauchbare Ergebnisse. Auch ist
oben vorgestellte Link-basierte Methode nicht in Stein gemeiselt; wie gesagt,
kann das Abstandmaß durchaus variiert werden, um sich besser den Daten
anzupassen, und auch das agglomerative Verfahren kann ein anderes sein.
So lässt sich k-Means dahingehend abändern, dass der "'Abstand"' zwischen
zwei Objekten der Anzahl ihrer Links, also gemeinsamen Nachbarn, entspricht.
Nicht zuletzt muss man sich auch Gedanken über die Repräsentation der
gefundenen Cluster machen. Im Allgemeinen können mit der Link-basierten
Methode auch nicht runde Cluster gefunden werden, insbesondere wären auch
stark sichelförmige Cluster möglich. Wie kann man aber solche Cluster ausgeben?
Der Mittelpunkt wird wohl nicht ausreichen und ist im Beispiel sogar irreführend.
Man wird wohl nicht darum herumkommen, mehrere Punkte ausgeben zu
müssen, damit der User ein Gefühl für das Cluster bekommt, aber das
dürfte gerade für hochdimensionale Lösungen wenig intuitiv sein.

\subsection{Dichte-basiertes Clustering}
\subsubsection{DBSCAN}
\textbf{Density-Based Spatial Clustering of Applications with Noise} (DBSCAN)
ist ein Verfahren, dass sich die \textbf{Dichte} von Clustern und
von unseren Datenobjekten zu Nutze macht. Dabei ist ein Objekt genau
dann dicht, wenn es mehr als \(minPts\) andere Datenobjekte innerhalb einer
\(\varepsilon\)-Umgebung besitzt. Punkte, die selbst in einer solchen
\(\varepsilon\)-Umgebung liegen, aber selbst nicht dicht sind, heißen
\textbf{dichte-erreichbar}.

Die Idee des Algorithmus ist nun Folgende: Wähle einen Punkte aus
dem Datenbestand. Prüfe, ob es dicht ist. Falls ja, dann ist der Punkt
ein potentieller Kandidat für ein Cluster und wir fügen alle seine Nachbarn
dem Cluster hinzu. Betrachte nun alle seine Nachbarn und prüfe, ob diese
dicht sind. Falls ja, wiederhole das Spiel von eben (d.h. expandiere die
Nachbarn um ihre \(\varepsilon\)-Umgebung und prüfe Dichte der neuen
Punkte). Wiederhole dies, bis alle Punkte erreicht wurden. Wähle dann
einen anderen, noch nicht besuchten Punkt aus. Wiederhole, bis alle Punkte
besucht wurden.

Man kann sehen, dass die Zuordnung von dichten Punkten deterministisch
ist. Nichtdeterministisch wird es, wenn wir dichte-erreichbarer Punkte
betrachten. Diese können auch zwischen zwei Clustern liegen. Bei ihrer
Zuordnung hängt es nun davon ab, welchen Cluster man zuerst erstellt.

Die Kosten dieses Verfahrens hängen maßgeblich von der Expansion, also
dem Finden der Nachbarn, ab. Ohne eine entsprechende Implementierung
müsste man jedes mal den ganzen Datenbestand scannen. Hierbei sind
räumliche Indexstrukturen ein wahrer Segen, z.B. R-Trees. Hat man eine
entsprechende Implementierung, so belaufen sich die Kosten etwa auf
\(O(n \log n)\). Im schlimmsten Fall hätten wir es stattdessen mit
\(O(n^2)\) zu tun.

Die \textit{Parameter} von DBSCAN sind zum einen \(\varepsilon\) und
zum anderen \(minPts\). Generell lässt sich sagen, dass eine Veränderung
der Parameter eine Auswirkung auf die \textit{Dichte} und
\textit{Größe} der gefundenen Cluster hat. Die Zahl der gefundenen Cluster
könnte auch variieren; das hängt jedoch sehr von dem gegebenen Datenbestand ab.
An und für sich ist es blöd, dass der Anwender diese Parameter spezifizieren 
muss, da sie sich stark auf die Qualität des Ergebnisses auswirken, aber
daran lässt sich nichts ändern. Um hier eine bessere Wahl der Parameter
zu ermöglichen, müsste Wissen über die Verteilung der Daten vorhanden
sein, was man nicht voraussetzen kann. Und wenn man dieses Wissen hätte,
wozu würden wir dann noch Clustern wollen? Man wird sich also insgesamt bei
der Parameterwahl nicht vor einem Trial \& Error Prozess drücken können.

\subsubsection{OPTICS}
\textbf{Ordering Points To Identify the Clustering Structure} (OPTICS) 
soll bei der Wahl des \(\varepsilon\) für unser DBSCAN helfen, indem es
die Auswirkungen auf das Clustering visualisiert für verschiedene Werte für
\(\varepsilon\). \(minPts\) muss jedoch noch weiterhin vom User selbst
gewählt werden.

Hierfür werden wir weitere Distanzkonzepte brauchen. Zum einen haben wir die
\textbf{core-distance}, die dem minimalen \(\varepsilon\) entspricht,
für das das Objekt noch \textbf{dicht} ist. Die andere Distanz ist die
\textbf{reachability-distance}\footnote{Auf Wikipedia wird's anders definiert.
Dort ist das "'undefined"' Kriterium, ob der aktuell betrachtete Punkt dicht ist,
oder nicht.}
von \(o\) zu Punkt \(p\)
\begin{align*}
 reach_{\bar{\varepsilon} , minPts}(o,p) = 
 \begin{cases}
 undefined & \text{wenn } dist(o,p) >\bar{\varepsilon},\\
\max \{dist(p,o), coreDist_{minPts}(o)\} & \text{sonst.}\\ % Mit großer Wahrscheinlichkeit ist coreDist_{minPts}(o) statt coreDist_{minPts}(p) korrekt. Siehe Vorlesung 2018-12-18 46:00 (https://opencast.informatik.kit.edu/engage/theodul/ui/core.html?id=50160409-5c46-4e75-b618-d7e9d7831bff)
\end{cases}
\end{align*}
Die Idee dahinter ist, dass wir uns zunächst nur für Objekte interessieren,
die überhaupt in der \(\bar{\varepsilon}\)-Umgebung unseres betrachteten
Objekte liegen. Dabei ist \(\bar{\varepsilon}\) der maximale Wert für
unser \(\varepsilon\). Indem wir also die nicht direkt erreichbaren Objekte
ersteinmal außer Acht lassen, haben wir die Hoffnung, dass wir nahe Objekte
auch möglichst zeitnah betrachten. Für die betrachteten Objekte gilt, dass,
wenn sie innerhalb des cores liegen, die Reihenfolge ihrer Betrachtung egal ist.
Das kommt daher, dass das Objekt für ein \(\varepsilon\) kleiner als seine
\(coreDist\) nicht mehr dicht wäre. Dann aber würde das Objekt auch nicht
mehr zu einem Cluster gehören. Insofern beschränken wir uns auf den
Fall, dass das Objekt doch dicht ist, also die Nachbarn des Objektes in
jedem Fall (unabhängig vom aktuellen \(\varepsilon\) betrachtet werden.
Dies gilt in der Form nicht für die Objekte, die außerhalb des cores liegen.
Diese sind nämlich nur dann vom aktuellen Objekt aus direkt erreichbar, wenn
das \(\varepsilon\) entsprechend gewählt wird. Der Unterschied zu den Punkten,
die außerhalb der \(\bar{\varepsilon}\)-Umgebung liegen, ist, dass die Punkte
\textit{überhaupt} erreicht werden können.\footnote{Irgendwie erscheint der
Ansatz von Wikipedia, Punkte auszuschließen, die nicht dicht sind, sinnvoller...}

Der Algorithmus geht wie folgt vor. Man wählt sich einen beliebigen Punkt
als Start. Von diesem berechnet man dann die core-distance. Seine 
reachability-distance ist am Anfang undefined, da wir die reachability-distance
für neue Punkte zu allen Punkten aus der Ausgabeliste berechnen, um so den
nächsten Punkt zu den bisher ausgegebenen zu finden. Da es aber noch keine 
Punkte in der Ausgabe gibt, wird der Wert einfach auf undefined gesetzt.
Als neuer Punkt wird dann einer der Punkte gewählt, die vom Startpunkt
aus erreichbar sind. Von diesen wird die reachability-distance berechnet und die
Punkte werden anhand derer in einer PriorityQueue gespeichert. Aus dieser
Queue werden dann die Objekte nacheinander ausgelesen, d.h. es wird immer
der nächste Punkte zur Ausgabeliste gewählt. Hat man einen neuen Punkt
gewählt, und besucht diesen (man fügt ihn also der Ausgabeliste hinzu), so
muss die PriorityQueue jedes Mal aktualisiert werden. Insgesamt erhält man
als Ausgabe also die Datenpunkte sortiert nach ihrer reachability-distance.

Plottet man dieses Ausgabe als Säulendiagramm, so dass man an der
Höhe die Größe der reachability-distance ablesen kann. Man stellt fest, dass
es immer wieder zu Tälern kommt, d.h. es gibt Mengen an Daten, die zueinander 
eine kleine reachability-distance haben. Das sind die Cluster. Unser variables
\(\varepsilon\) kann man sich nun als horizontale Linie vorstellen, die man
über den Plot legen und vertikal verschieben könnte. Alle Täler, die für ein
bestimmtes \(\varepsilon\) noch unter der Horizontalen liegen, sind für dieses
\(\varepsilon\) dicht, d.h. sie sind Kandidaten für Cluster. Im Allgemeinen muss
der User die im Plot dargestellten Cluster noch von Hand untersuchen, um
festzustellen, welche dieser sich tatsächlich als tauglich erweisen.

Die Laufzeit ist im Grunde ähnlich wie bei DBSCAN; da jedoch unser
\(\bar{\varepsilon}\) fast immer größer ist als das \(\varepsilon\) aus
DBSCAN, sind die Nachbarschaftsanfragen bei OPTICS aufwändiger.

Die Parameter sind dieses Mal das \(\bar{\varepsilon}\) und \(minPts\).
Veränderungen von ersterem führen nur zu relativ kleinen Veränderungen.
Man wählt diesen Parameter sowieso etwas größer, damit man später
Variationsmöglichkeiten bei der Wahl des \(\varepsilon\)-Wertes hat. Für
genügend große Werte von \(\bar{\varepsilon}\) gibt es also kaum Probleme.
Wählt man den Wert aber zu klein, dann ergeben sich daraus kleinere Cluster
mit höherer Dichte. Große Cluster, die jedoch eine geringere Dichte besitzen
bleiben außen vor. Warum würde man dann nicht einfach das \(\bar{\varepsilon}\)
riesig wählen, sodass immer alle Objekte in der \(\bar{\varepsilon}\)-Umgebung 
liegen? So würde man das "'Full-Picture"' bekommen. Das Problem ist hierbei
die Performance. Die Laufzeit würde unweigerlich quadratisch werden, selbst
bei geeigneten Implementierungen, da nun in jedem Durchlauf alle Datenobjekte
gescannt und geupdatet werden müssten. Der andere Parameter \(minPts\)
wirkt sich vor allem auf notwendige Dichte, und damit auch auf die Zahl der Cluster
aus. Wählt man ihn zu klein, dann hat man das Problem, dass die Cluster zu
kleinteilig werden, d.h. die Darstellung sieht das sehr zersplittert aus, weil sich
große Cluster in eine Unmenge ein kleinen Unterclustern aufteilen. Eine 
gute Wahl von \(minPts\) dürfte wohl ebenfalls erst nach einem Trial \& Error
Prozess möglich sein.

\subsection{Probabilistisches Clustering}
Wie bereits bei der Klassifikation angesprochen ist es meistens nicht wirklich
sinnvoll (und evtl. auch gar nicht möglich) Objekte zu 100\% einem
bestimmten Cluster zuzuweisen. Stattdessen will man die Objekte
mit einer bestimmten Wahrscheinlichkeit einem Cluster zuordnen. Die setzte 
aber voraus, dass wir Annahmen über die Verteilungen innerhalb der Cluster
und auch über die Auftrittswahrscheinlichkeit eines Clusters treffen.
Dieses Problem wird auch \textit{Finite Mixture Problem} genannt.

\subsubsection{Expectation Maximization}
Ein Algorithmus, der uns hierbei hilft, ist der \textbf{EM-Algorithmus}.
Die Idee ist es, einen Art ML-Schätzer (Maximum-Likelihood-Schätzer) auf unsere angenommenen
Verteilungen anzuwenden und die Parameter dahingehen anzupassen,
dass sich die angenommenen Verteilungen möglichst gut mit den echten
decken.

Nehmen wir dafür zunächst eine Normalverteilung in allen Clustern an. Dann
raten wir für alle Cluster deren \(\mu\) und \(\sigma\) und 
für die Cluster als solche raten wir die jeweilige Wahrscheinlichkeit für
ihren auftritt \(p_k\). Was wir nun berechnen
wollen, ist die Wahrscheinlichkeit, dass ein gegebenes Objekt zu einem
bestimmten Cluster gehört, also
\[
P[Cluster_k \mid x] = \frac{P[x \mid Cluster_k] \cdot P[Cluster_k]}{P[x]}
=\frac{f(x, \mu_k,\sigma_k) \cdot p_k}{P[x]}\text{,}
\]
wobei \(x\) ein konkretes Datenobjekt sei und \(f(\mu,\sigma,x)\) die
Wahrscheinlichkeitsdichtefunktion des Clusters \(k\) sei. Auf der rechten Seite
fehlt eigentlich nur noch das \(P[x]\) zur Berechnung des Quotienten. Es gilt
jedoch auch \(P[x] = \sum_k P[x \mid Cluster_k] \cdot p_k\). Das ist genau der
Teil, der im Zähler unserer obigen Gleichung steht. Und da wir über alle Cluster
aufsummieren, bedeutet das effektiv eine Normalisierung unserer errechneten
Werte. Die Bestimmung dieser Wahrscheinlichkeiten ist der \textbf{Expectation
Schritt}, als nächstes kommt der \textbf{Maximization Schritt}. Hierbei wollen
wir unsere geratenen Parameter möglichst den echten Werten annähern. Hierfür
berechnen wir für jedes Cluster
\begin{align*}
\mu_k = \frac{\sum_{i=1}^{|DB|}w_i x_i}{\sum_{i=1}^{|DB|} w_i} &\qquad &
\sigma_k = \frac{\sum_{i=1}^{|DB|} w_i(x_i - \mu)^2}{\sum_{i=1}^{|DB|} w_i}.
\end{align*}
Dabei sei \(|DB|\) die Mächtigkeit unseres Datenbestandes und \(w_i\) entspricht
unserem im Expectation Schritt errechneten \(P[A \mid x]\). Damit haben wir
unsere neuen Werte für den nächsten Durchlauf, d.h. die vorhin geratenen
Werte werden nun für den kommenden Expectation Schritt durch die eben
errechneten Werte ersetzt.\footnote{Tatsächlich wurde in der Vorlesung
nur vorgestellt, wie die Parameter der Verteilungen \textbf{innerhalb} der Cluster
neu berechnet werden. Die Annahme über die Verteilung der Cluster selbst
(\(p_k\)) bleibt fest. Das dürfte wohl insofern kritisch sein, als dass man
schon im Vorhinein ein Wissen über die Wahrscheinlichkeiten der Cluster haben
muss. Das dürfte jedoch i.A. nicht der Fall sein...}
Die beiden Schritte werden dann so lange wiederholt, bis sich unsere
\textit{Terminierungsfunktion} nicht mehr wesentlich ändert. Sie ähnelt einem
ML-Schätzer\footnote{Eigentlich fast identisch bis auf die Gewichtung mit den
\(p_k\), die der Verfasser als kritisch sieht. Ein einfacher ML-Schätzer, wie er
im Rest der Literatur verwendet wird, dürfte ein genauso gutes Ergebnis liefern 
ohne Voraussetzung von Domänenwissens.}
und lautet
\[ \prod_{i=1}^{|DB|}\left( \sum_{k=1}^{|Cluster|} 
f(x_i,\mu_k,\sigma_k) \cdot p_k \right). \]
Diese Funktion soll maximiert werden. Es lässt sich beweisen, dass sie auch immer
konvergiert, weswegen unsere Iterationen dann abgebrochen werden sollten, 
wenn für eine bestimmte Zahl an Durchläufen sich der Wert unserer
Terminierungsfunktion nicht mehr signifikant verändert. Diese Funktion ist nicht
robust gegen Ausreißer: Sollte ein Wert nur eine sehr kleine Wahrscheinlichkeit besitzen,
in überhaupt irgendein Cluster zu fallen, dann wird der entsprechende Faktor in
obigem Produkt sehr sehr klein, was Auswirkungen auf den gesamten Wert hat. 
Das ist jedoch zum einen aufgrund der Konvergenz eigentlich kein Problem, am
Ende kommen wir doch an das Ziel. Zum anderen ließe sich wie beim 
herkömmlichen ML-Schätzer stattdessen die log-Likelihood berechnen, wodurch
die Funktion weniger anfällig für Ausreißer wird, und die Werte, mit denen man
rechnet, werden auch schöner.

Dieser Algorithmus findet nur lokale Optima, es müssen also mehrere Runs
mit verschieden geratenen Parametern durchgeführt werden.

\subsubsection{Erweiterungen von Mixture Models}
In der Vorlesung wurden folgende Erweiterungen von Mixture Models vorgestellt:
\begin{itemize}
\item \textbf{Mehr als 2 Klassen:} Das haben wir oben bereits getan, in der
Vorlesung wurde alles nur anhand von 2 Klassen präsentiert. Wir machen das hier 
aus Leidenschaft natürlich gleich explizit für den allgemeinen Fall.
\item \textbf{Mehr als ein Attribut:} Hier kommt es darauf an, ob wir von
Unabhängigkeit der Attribute ausgehen, oder nicht. Falls sie unabhängig sind,
ist der Fall eigentlich ganz einfach: Aus unseren \(P[x \mid A]\) wird 
\(\prod_{i=1}^{|I|} P[x_i \mid A]\), wobei \(I\) die Anzahl der
Attribute ist. Das Problem hierbei ist jedoch die
Anzahl an Parametern. Wir müssten jetzt für jedes Cluster \(I\) Mittelwerte
und \(I\) Varianzen berechnen. Das ist ein bisschen viel.\\
Gehen wir aber nun stattdessen von abhängigen Variablen aus, dann wird 
der Aufwand sogar noch größer: Wir müssen nun nicht mehr die Standardabweichungen
berechnen, sondern die Kovarianz-Matrix. Diese ist quadratisch mit \(I^2\) vielen
Einträgen. Aufgrund der Symmetrie reicht es jedoch, nur eine Dreiecksmatrix davon
zu berechnen. Damit ergeben sich für Cluster \(I + \frac{I(I-1)}{2}\) viele 
Parameter (Standardabweichungen + Einträge Kovarianz-Matrix).
\item \textbf{Kategorische Attribute:} Bisher sind wir nur von numerischen
Attributen ausgegangen, d.h. die Berechnung der Mittelwerte und 
Standardabweichungen ist relativ "'straight forward"'. Diesem Teil werden wir uns unten
widmen.
\item \textbf{Andere Verteilungen:} Da wir im Prinzip einen ML-Schätzer als
Gütemaß verwenden, können auch einfach andere Verteilungen verwendet werden.
Kleine Anpassungen sind nötig, der Algorithmus läuft im Groben aber gleich ab.
Die Wahl der Verteilungen setzt aber ebenfalls zumindest ein grobes Wissen über 
die gegebenen Daten voraus. Wählt man die Verteilungen jedoch passend, so 
kann man eine gute Darstellung des Datenbestandes erreichen.
\end{itemize}
Wie eben angekündigt wollen wir uns jetzt nochmal mit dem Thema kategorische
Daten beschäftigen. Hier ist das Vorgehen auf den ersten Blick nicht ganz klar, da
wir nicht von einer Normalverteilungsannahme ausgehen können. Betrachten wir
den Fall für \textbf{ein} Attribut, das \(\nu\) Ausprägungen annehmen kann
(z.B. Ausprägung des Attributs Marke einer Kamera wäre Nikon, Canon, Sony,
 \(\dots\). Dann haben wir bei \(k\) Clustern insgesamt \(k \cdot \mu\) Parameter.
 Wir müssen nämlich pro Cluster die Wahrscheinlichkeit für das Auftreten der
 Merkmale schätzen (der Verfasser stellt sich dies als Tortendiagramm pro Cluster
 vor). Die Wahrscheinlichkeit der Clusterzugehörigkeit bei einem gegebenen 
 Datenobjekt ist dann
 \[ P[Cluster \mid x] = \frac{P[x \mid Cluster] \cdot P[Cluster]}{P[x]}.\]
 Die ist unser Expectation Schritt. Im ersten Schritt raten wir dafür 
 \(P[x \mid Cluster]\) und \(P[Cluster]\). Im Maximization Schritt wollen
 wir diese Parameter berechnen. Hierfür berechnen wir erst
 \[ P[Cluster] = \sum_i P[Cluster \mid x_i] \cdot P[x_i].\]
 Hierbei nutzen wir die eben berechneten Wahrscheinlichkeiten und die 
 Werte für \(P[x_i]\) lassen sich direkt durch Abzählen bestimmen, sind
 also die relativen Häufigkeiten. Mit diesem Wert lässt sich nun noch das 
 fehlende
 \[P[x \mid Cluster] = \frac{P[Cluster \mid x] \cdot P[x]}{P[Cluster]}\]
 berechnen. Diese Werte werden nun für die nächsten Iterationen verwendet.
 
 Als Terminierungsfunktion dient nun wieder eine Art ML-Schätzer:
 \[\prod_{i=1}^{|I|}\left( \sum_{k=1}^{|K|}
 P[x \mid Cluster] \cdot P[Cluster] \right)\]
 Es kann jedoch aber auch vorkommen, dass ein Attributwert gar nicht
 angenommen wird. Dann wird \(P[x] = 0\) und wir hätten an einigen
 Stellen in unserer Berechnung ein Problem. Deswegen wird mithilfe des
 \textbf{Laplace Estimators} einfach das Vorkommen eines jeden Merkmals um
 1 erhöht. Man tut so, als würde die Merkmalsausprägung wenigstens einmal
 vorkommen.
 
 Auch hier wird nur ein lokales Optimum gefunden, d.h. mehrere Runs sind nötig.

 Im Falle von mehreren Attributen ist das Vorgehen sowohl im korrelierten als auch
 im unabhängigen Fall gleich: Man konstruiert aus den \(\i\) Attributen mit % Möglicherweise \(I\) gemeint
 jeweils \(\nu_i\) Ausprägungen ein neues Attribut, welches \(\prod_i \nu_i\) 
 Ausprägungen hat. Dabei dürfen die selbst Attribute jedoch nur überschaubar
 in ihrer Anzahl sein und sie dürfen jeweils nicht zu viele Ausprägungen besitzen.
 Die Zahl der zu schätzenden Parameter wäre schlicht zu groß. Außerdem besteht
 dabei umso mehr die Gefahr des Overfittings. 
 
 Im Übrigen lässt sich festhalten, dass es zwar Lösungen für Datenbestände gibt,
 die numerische und kategorische Daten zusammen verarbeiten können, diese
 jedoch oft sehr komplex sind.
 
 Zusammenfassend noch ein paar Punkte bzgl. EM und Overfitting. EM setzt darauf,
 dass der User die Anzahl der Cluster und die Korrelationen der Attribute
 explizit angibt. Generell gilt, zu viele Parameter führen gerne mal zu Overfitting.
 Deswegen sollten nicht alle Attribute als kovariat markiert werden. Auch muss
 bei der Wahl der Anzahl der Cluster Vorsicht walten; zu viele Cluster und es
 entsteht ein Overfitting (worst case wäre Anzahl der Cluster = Anzahl der
 Datenobjekte). Auch sollten Verteilungen mit zu vielen Parametern vermieden werden,
 da diese zum einen mehr Rechenaufwand erfordern können und zum anderen
 leicht zu Overfitting führen, da man durch viele Parameter "'Tweaks"' leichter
 ein Clustering finden kann, dass "'gut aussieht"', aber zu sehr an den Trainingsdaten
 orientiert ist.